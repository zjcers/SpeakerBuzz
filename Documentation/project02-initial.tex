\documentclass{article}

\title{Audio BoosterPack Proposal}
\author{Zane J Cersovsky}
\date{September 30, 2017}

\begin{document}
    \maketitle
    \paragraph{First Lab}
    The very first thing to do would be something akin to the current SpeakerBuzz lab. While this lab wouldn't offer 
    much in terms of complexity, it is a solid place to start. This lab would consist of bringing the necessary hardware
    online and then generating a constant frequency tone. Like the current SpeakerBuzz lab, it would intentionally suffer 
    from the interference of the other tasks (since this would not be using FreeRTOS).

    \paragraph{Second Lab}
    In the second lab, I would like to introduce the notion of interrupts and hardware timers. It would be based on the 
    first lab, but would move the logic for generating the tone into a timer ISR. This would be a fairly easy introduction,
    in my opinion, to both timers and interrupts since it maps directly to the timer that would be kept in "userspace" in 
    the first lab.

    \paragraph{Third Lab}
    Lastly, I would like to do a lab utilizing the FPU on the Cortex M4. Following the pattern above, it would be based on the 
    previous lab. The difference is that instead of a constant tone, this lab would involve doing a frequency sweep. I have yet 
    to decide between which to frequencies, but that is mostly a function of finding a wide-enough range to be meaningful while 
    not giving the TA headaches. Furthermore, it would probably be a good idea to have the sweep either repeat or reverse so 
    that the program never actually terminates.

\end{document}