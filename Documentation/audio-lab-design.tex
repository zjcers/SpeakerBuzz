\documentclass{article}
\usepackage{tabularx}
\title{Speakerbuzz and beyond}
\author{Zane Cersovsky}
\begin{document}
\maketitle
\paragraph{Observations and Current Knowledge}

The Audio Boosterpack has a Serial Peripheral Interface-driven Digital Audio Converter and an amplifier.
The first item is concerning since Tivaware only has built in drivers for its Synchronous Serial Interface.
The SSI peripheral can implement SPI, but since the driver is generic for several different protocols, so 
a more specific SPI abstraction should be constructed. Secondly, the amplifier may be configurable or require
some sort of initialization.

\paragraph{Design}
\subparagraph{Overview}
As mentioned previously, an abstraction (driver) for SPI should be implemented. One possible design 
would be to emulate the interface of the I2C driver in Tivaware sensorlib. Barring jumping to embedded C++,
this may be one of the cleaner ways to implement this. Similarly, a thin layer should be implemented over 
the DAC. This layer should be able to initialize, reconfigure, and write to the DAC through the SPI driver. 
Over top of this, a driver for the entire Boosterpack should encapsulate at least the DAC and amplifier, if 
the amplifier requires configuration. In the future, other components or related functionality could be 
integrated here. None of these drivers should worry about concurrency since it is possible that various 
parts of them could be driven by interrupts, a main loop, or a real time operating system.

\subparagraph{SPI driver}
The first step will be to build the SPI driver. Taking a hint from the TI sensorlib drivers, the actual initialization
of the ports will be left to the user since this driver should not "own" the hardware it controls. (Though it would be 
handy to provide a method to configure the given SSI device to work in SPI mode.) This driver will be a simple wrapper 
around the SSI driver in TI driverlib. As mentioned previously, this driver leaves concurrency up to the programmer. \\

\begin{tabularx}{\linewidth}{| X | X |}
    \hline
    Method Name & Description \\
    \hline
    \texttt{SPIInit(SPIInstance, SSIDevice)} & Initializes the SPIInstance data structure and the given SSI device. \\
    \texttt{SPIWrite(SPIInstance, Buffer, BufferSize)} & Writes the buffer to the device \\
    \hline
\end{tabularx}

\subparagraph{DAC driver}
This driver will handle communications with a TI DAC8311 at a higher level, using the SPI driver. While it would be 
possible for this driver to be able to play sounds (by configuring a timer interrupt), this will not be provided since that is 
the point of the labs. What it will provide are methods to configure the DAC and write values to it. \\

\begin{tabularx}{\linewidth}{| X | X |}
    \hline
    Method Name & Description \\
    \hline
    \texttt{DACInit(DACInstance, SPIInstance)} & Initializes the DACInstance data structure and the connected DAC \\
    \texttt{DACWrite(DACInstance, Data)} & Writes a single value to the DAC \\
    \hline
\end{tabularx}

\subparagraph{Boosterpack driver}
Lastly, the Boosterpack driver will initialize the previous two drivers and configure them for use with an Audio Boosterpack. It 
will also handle the configuration of other devices on, or related to the Boosterpack (such as the amplifier). The DAC writing 
functionality will be made available in a simplified form so that it will be easy to use in the labs but not encroach on the labs
 themselves.

\paragraph{Exploration}
\subparagraph{Overview}
Future question: Can I clock frames out of the SSI with a timer or timer interrupt? This would be useful since the process would 
only have to refill the hardware FIFO every 8 frames (the size of the FIFO).

\subparagraph{SSI Pinout}
\begin{center}
    \begin{tabular}{| r | r | r |}
        \hline
        SSI Device & Function & Pin \\ 
        \hline
        0 & Clock & PA2 \\
        & I/O 0 & PA4 \\
        & I/O 1 & PA5 \\
        & I/O 2 & PA6 \\
        & I/O 3 & PA7 \\
        \hline
        1 & Clock & PB5 \\
        & I/O 0 & PE4 \\
        & I/O 1 & PE5 \\
        & I/O 2 & PD4 \\
        & I/O 3 & PD5 \\
        \hline
        2 & Clock & PD3 \\
        & I/O 0 & PD1 \\
        & I/O 1 & PD0 \\
        & I/O 2 & PD7 \\
        & I/O 3 & PD6 \\
        \hline
        3 & Clock & PQ0 or PF2 \\
        & I/O 0 & PQ1 or PF1 \\
        & I/O 1 & PQ3 or PF0 \\
        & I/O 2 & PF4 or PP0 \\
        & I/O 3 & PP1\\
        \hline
    \end{tabular}\\
This was retrieved from Table 17-1 of the TM4C1294NCPDT datasheet.
\end{center}

\subparagraph{Audio Boosterpack Pin Mapping}
\begin{center}
    \begin{tabular}{| r | r | r | r |}
        \hline
        Tiva Pin & Tiva Function & Boosterpack Pin & Function \\
        \hline
        & & A2-1 & Vcc \\
        PD4 or PA0 & AIN7 & A2-5 & MIC power \\
        PD5 or PA1 & AIN6 & A2-6 & MIC output \\
        PQ0 & SSI3 clk & A2-7 & DAC SCLK \\
        PQ1 & SSI3 frame signal & D2-9 & DAC SYNC \\
        PP3 & GPIO & D2-8 & AMP on \\
        PQ2 & SSI3 data 0 & D2-6 & DAC MOSI \\
        PM7 & PWM & D2-2 & PWM \\
            & & D2-1 & GND \\
        \hline
    \end{tabular}\\
This was derived from the Tiva Eval board datasheet in combination with the Audio Boosterpack User Guide.
\end{center}

\paragraph{Implementation Notes}

Not implemented yet.

\paragraph{Lab 1}
\subparagraph{Overview}
This lab is similar to the current Speakerbuzz, but reimagined for the Tiva and Audio Boosterpack. A single predefined tone will 
be emitted with the timing coming from vTaskDelay. This will provide unreliable timing, which is part of the object of the lab.

\subparagraph{Hardware and Software Utilized}
This lab will use all of the drivers that I will have to implement. These drivers then use the SSI and GPIO drivers from driverlib. 
The hardware used includes SSI3 which needs Tiva ports Q and P enabled. FreeRTOS will be used for my convenience, but is not 
strictly required since this could be readily adapted for the current task-loop that the early 388 labs use.

\subparagraph{Program Structure}
In the entry method (main), all of the drivers will be initialized and made available. A single task, Speakerbuzz, will look like 
this: \\
\begin{verbatim}
Begin Task
    DelayAmount := SysTickFreq/2*frequency
    Volume := DACMaxVal/2
    TopOfWaveform := false
    Loop Forever
        If TopOfWaveform
            DACWrite(Volume)
        Else
            DACWrite(0)
        End If
        TopOfWaveform = Not TopOfWaveform
        Delay for DelayAmount
    End Loop
End Task
\end{verbatim}
\paragraph{Lab 2}
\subparagraph{Overview}
This lab will build on top of the last one, but this time the timing will be derived from a timer interrupt instead of vTaskDelay 
or check against SysTick. Once again, FreeRTOS will be used, but is not required. The point of this lab is to learn more about 
interrupts.
\subparagraph{Hardware and Software Utilized}
Like the previous lab, all of the implemented drivers plus SSI and GPIO will be used. In addition, a hardware timer and 
corresponding interrupt will be configured. The interrupt handler will perform the same function as the task in the previous lab.
\subparagraph{Program Structure}
\begin{verbatim}
Begin Init
    Init Boosterpack
    Init Timer
    Register Interrupt

    TopOfWaveform := false
    Volume := DACMaxVal/2
End Init
Begin Interrupt Handler
    If TopOfWaveform
        DACWrite(Volume)
    Else
        DACWrite(0)
    End If
    TopOfWaveform = Not TopOfWaveform
End Interrupt Handler
\end{verbatim}
\paragraph{Lab 3}
\subparagraph{Overview}
Lastly, this lab builds on the last by have a separate task that changes the tone frequency along some function that requires the 
FPU. An a example would be the function $e^x$ with the output stretched to be a nice frequency range (such as 500hz to 2000hz) and 
$x$ reset to zero when it reaches the maximum frequency. Again, FreeRTOS is optional.
\subparagraph{Hardware and Software Utilized}
The modules and hardware from Lab 2 would be required since this builds on top of that lab. In addition, the FPU would need to 
be enabled and a new task written to perform the frequency sweep.
\subparagraph{Program Structure}
\begin{verbatim}
Begin Task FreqSweep
    MinXVal := "some low number, probably zero"
    MaxXVal := "some higher number where the frequency max is"
    Increment := "amount to increment x for each step"
    IncrementDelay := "amount of time between each increment"
    X := MinXVal
    Loop Forever
        If X >= MaxXVal
            X = MinXVal
        End If
        Frequency := Func(X)
        TimerDelay := ToTicks(Frequency)
        X += Increment
        Delay(IncrementDelay)
    End Loop
End Task
\end{verbatim}
\end{document}